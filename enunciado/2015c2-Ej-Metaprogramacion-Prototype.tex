\documentclass[spanish,a4paper]{article}
\usepackage{hyperref}
\newcommand{\version}{final} % final, interna o con soluciones (ojo escribirlo 'tal cual').
\newcommand{\cuatrimestre}{2do cuatrimestre 2015} 
%%!TEX encoding = UTF-8 Unicode 
\usepackage[utf8]{inputenc}
\usepackage{amstext}
\usepackage{ifthen}
\usepackage{graphicx}
\usepackage{array}
\usepackage{xspace}
\usepackage{color}
\usepackage{fancyvrb}
\usepackage{version}
\usepackage{theorem}
\usepackage{listings}

\usepackage[spanish]{babel}

\textwidth 17.5cm
\textheight 26cm
\oddsidemargin -1cm
%\evensidemargin -.3cm
\topmargin -1cm
\headheight 0cm
\headsep 0cm

%\textwidth 16cm
%\textheight 25cm
%\oddsidemargin -.64cm
%\evensidemargin -.64cm
%\topmargin -1.5cm
%\headheight 0cm
%\headsep 0cm
\definecolor{gray}{rgb}{0.5,0.5,0.5}
\lstset{ %
  backgroundcolor=\color{white},  % choose the background color. You must add \usepackage{color}
  showspaces=false,               % show spaces adding particular underscores
  showstringspaces=false,         % underline spaces within strings
  showtabs=false,                 % show tabs within strings adding particular underscores
  frame=single,	                % adds a frame around the code
  tabsize=2,	                % sets default tabsize to 2 spaces
  captionpos=true,                   % sets the caption-position to bottom
  breaklines=true,                % sets automatic line breaking
  breakatwhitespace=true,        % sets if automatic breaks should only happen at whitespace
  escapeinside={\%*}{*)/},         % if you want to add a comment within your code
  aboveskip=3mm,
  belowskip=3mm,
  columns=flexible,
  basicstyle=\small\ttfamily,
  numberstyle=\tiny\color{gray},
  keywordstyle=\bfseries,
  commentstyle=\color{gray},
  stringstyle=\color{gray},
}

\lstdefinelanguage{smalltalk}{
  morekeywords={self,super},
  otherkeywords={:=,^,\[,\],>>,\(,\),\;},
  sensitive=true,
  morecomment=[n]{"},
  morestring=[b]'
}

\lstloadlanguages{Ruby}
\lstset{%
  basicstyle=\ttfamily\color{black},
  commentstyle = \ttfamily\color{red},
  keywordstyle=\ttfamily\color{blue},
  stringstyle=\color{orange}
}

\renewcommand{\thepage}{}
\newcommand{\ignore}[1]{}
\newcommand{\mat}[1]{\ensuremath{#1}}

\newenvironment{display}
   {\begin{list}{}{\setlength{\topsep}{0cm}
                   \setlength{\leftmargin}{0cm}
                  }}{\end{list}}

\newcommand{\textfol}[1]{\ensuremath{\textsf{#1}}}
\newcommand{\textcode}[1]{{\normalfont{\texttt{#1}}}}
\newcommand{\propername}[1]{\textsc{#1}}

%\renewcommand{\labelenumi}{\textbf{\alph{enumi})}}
\renewcommand{\labelenumii}{\textbf{\alph{enumii})}}

\newcommand{\pred}[2]{\ensuremath{\text{\textbf{#1}}(#2)}}
\newcommand{\predName}[1]{\ensuremath{\text{\textbf{#1}}}}
\newcommand{\var}[1]{\ensuremath{#1}}
\newcommand{\code}[1]{\texttt{#1}}

\newcommand{\paratodo}[2]{\ensuremath{(\forall{#1})#2}}
\newcommand{\existe}[2]{\ensuremath{(\exists{#1})#2}}
\newcommand{\implicasp}{\ensuremath{\Rightarrow}}
\newcommand{\ysp}{\ensuremath{\,\wedge\,}}
\newcommand{\osp}{\ensuremath{\,\vee\,}}
\newcommand{\nosp}{\ensuremath{\neg}}
\newcommand{\implica}{\ensuremath{\Rightarrow}}
\newcommand{\y}{\ensuremath{\,\wedge\,}}
\renewcommand{\o}{\ensuremath{\,\vee\,}}
\newcommand{\no}{\ensuremath{\neg}}

\newcommand{\agujero}[1]{\ensuremath{<}\textit{#1}\ensuremath{>}}
\newcommand{\vble}[1]{\ensuremath{#1}}
\newcommand{\mkP}[2]{\ensuremath{p_{{#1#2}}}}

%%%%%%%%%%%%%%%%%%%%%%%%%%%%%%%%%%%%%%%%%%%%%%%%%%%
% Manejo de versiones (final o interna)
%%%%%%%%%%%%%%%%%%%%%%%%%%%%%%%%%%%%%%%%%%%%%%%%%%%

\newcommand{\final}[1]{\ifthenelse{\equal{\version}{interna}}{}{#1}}
\newcommand{\nofinal}[1]{\ifthenelse{\equal{\version}{interna}}{#1}{}}

\newcommand{\consolu}[1]{\ifthenelse{\equal{\version}{con soluciones}}{#1}{}}
\newcommand{\sinsolu}[1]{\ifthenelse{\equal{\version}{con soluciones}}{}{#1}}

\newboolean{final}
\final{\setboolean{final}{true}}
\nofinal{\setboolean{final}{false}}

\newcommand{\internal}[1]{ 
 \nofinal{\begin{flushleft} \textcolor{blue}{\upshape #1} \end{flushleft}}
}

\newcommand{\comentario}[1]{\textbf{\internal{#1}}}
\newcommand{\objetivo}[1]{\internal{\textbf{Objetivo:} #1}}

\sinsolu{\excludeversion{solucion}}
\consolu{
	\newcommand{\solucionTitle}{{\normalfont\textbf{\textcolor{blue}{Solución posible:}}}}
	\DefineVerbatimEnvironment{solucion}{Verbatim}
		{fontshape=n,tabsize=0,fontsize=\small,xleftmargin=16pt,formatcom=\solucionTitle\color{blue}}
}

\DefineVerbatimEnvironment{ejemplo}{Verbatim}{fontshape=n,tabsize=0,fontsize=\small,xleftmargin=16pt,}


\newcommand{\formalidadesPrimerEnunciado}{
  \section{Objetivos y forma de entrega}
  
  \subsection{Objetivos generales}
    El trabajo práctico debe realizarse en grupos de 5 personas. El trabajo práctico tiene como objetivo garantizar 
    que durante el transcurso de la materia todos los estudiantes pasen por la experiencia de:

    \begin{itemize}
      
      \item participar en un diseño complejo,
      \item bajar a código sus ideas de diseño en una computadora y verificar su funcionamiento mediante la utilización 
      de test cases.
      \item participar de una experiencia de trabajo grupal basada en herramientas de colaboración de tipo industrial, en
      particular compartiendo el código mediante un repositorio de fuentes compartido.
    \end{itemize}

    Para garantizar esto habrá tareas individuales y grupales. La entrega global deberá ser una sola para todo el 
    grupo, pero identificando en donde se solicite las responsabilidades que tomó cada uno. De esta forma se logra 
    un ambiente de trabajo similar al profesional, en el que cada uno tiene sus responsabilidad individuales sobre un
    proyecto compartido y el éxito del proyecto depende del trabajo conjunto.

    La evaluación del tp tendrá dos calificaciones, una grupal y otra individual. En caso de que alguno de los 
    integrantes del grupo no demuestre haberse comprometido con el trabajo grupal se le realizará un coloquio en el
    que deberá defender la solución propuesta por él y por el grupo.

    En la primera etapa del tp tendrán preponderancia las tareas grupales, con asignaciones individuales muy
    pequeñas, mientras que en la segunda parte esto se invertirá, para tener una mayoría de responsabilidades repartidas
    en asignaciones individuales.

    \subsection{Interacción con el tutor}
      Cada grupo tendrá asignado un tutor que los guiará no sólo para ir encontrando la mejor solución posible sino
      también en cuanto al proceso de trabajo, \textbf{ambos aspectos son igualmente importantes para la materia} y
      serán tenidos en cuenta para decidir la aprobación o no del trabajo práctico.

      Consideramos muy importante interactuar frecuentemente con el tutor y no esperar a la fecha de entrega para
      realizar las consultas. \textbf{No se aceptarán entregas fuera de término}, ya que esto es contrario a las necesidades
      del aprendizaje, las fechas no están puestas por capricho sino para ayudarlos a asimilar los conocimientos de la
      mejor manera posible. Si un grupo o un integrante no puede asistir al día de la entrega deberá responsabilizarse
      de coordinar una alternativa con el tutor, \textbf{en forma previa a la fecha límite}.

    \subsection{Forma de entrega}
      Para realizar el TP y almacenar tanto el código fuente como los documentos se deberá crear un área de trabajo
      compartida basada en un repositorio con capacidad de versionado, por ejemplo se pueden utilizar los repositorios
      \emph{svn} provistos en \texttt{www.xp-dev.com}.

      En cada entrega se deberá contar con el código en el aula para poder validarla. Quienes tengan la posibilidad de
      traerla en su propia notebook podrán hacerlo, en caso contrario deberán coordinar con el tutor para que él pueda
      tener el código en su propia notebook o el mecanismo alternativo que decidan en conjunto con el tutor. En caso
      que no contar con el código para evaluarlo en el momento de la entrega, la misma se considerará \emph{desaprobada}.
      
      La entrega constará del código fuente que resuelva los requerimientos y la documentación en papel correspondiente que
      explique las decisiones de diseño tomadas para llegar a dicha solución. Lo interesante de esta documentación no es una
      explicación del código (eso se puede ver allí) sino las alternativas que consideraron y los motivos por los cuales eligieron
      la solución que se implementó.


  \section{Requerimientos del Sistema}

}



\newcommand{\formalidadesSegundoEnunciado}{
 \section{Objetivos y forma de entrega}
 
   En esta siguiente parte del trabajo práctico se presentan 5 nuevos requerimientos, que se agregan a lo ya realizado
   para la anterior parte. Estos nuevos requerimientos deberán repartirse entre los integrantes del grupo, uno para
   cada uno. En los casos en los que el grupo tenga menos de 5 integrantes se podrá dejar afuera algún requerimiento,
   previo acuerdo con el tutor del grupo.

   Cabe recordar que no por esto la entrega deja de ser grupal. La entrega del TP debe ser una sola, pero
   identificando las responsabilidades asumidas por cada uno. Esto tiene dos objetivos: el primero es acercarse a la
   forma de colaboración que uno tendría en un proyecto laboral; el segundo es garantizar que todos trabajen, pero
   sin perder la posibilidad de interactuar con los compañeros.

   De esta forma cada uno deberá proponer el diseño que soluciona la tarea que le fue asignada, pero al mismo
   tiempo deberá consensuar con los demás integrantes del grupo las modificaciones que quiere hacer sobre el diseño
   base.
    \smallskip

    Para la entrega, deberán traer la documentación impresa actualizada que explica las decisiones de diseño que llevaron
    a la solución actual, así como todo el código terminado. Lo importante del documento impreso es que explique cuales
    fueron las decisiones de diseño que se tomaron para llegar a la solución. La implementación debe incluir todos los tests cases que sean necesarios que sean necesarios y que muestren el uso del sistema.

    En todos los casos nos planteamos como objetivo resolver las correcciones del TP en el día para ello deberán
    coordinar con el tutor algún mecanismo para tener a mano el código y poder verlo en conjunto.


  \section{Nuevos requerimientos}
 
}

\newcommand{\formalidadesTercerEnunciado}{
 \section{Objetivos y forma de entrega}
 
  El objetivo de esta entrega es construir un DSL utilizando el lenguaje Groovy para permitir la creación de reuniones según lo
  especificado en la siguiente sección. Es un requerimiento básico que el DSL generado interactúe con la solución desarrollada en Java
  en las entregas anteriores.
  Es importante también tratar de lograr que el DSL creado sea lo más expresivo posible considerando al menos los elementos del
  lenguaje vistos en clase.
  \smallskip
  
  Nos planteamos como objetivo resolver las correcciones del TP en el día para ello deberán coordinar con el tutor algún mecanismo
  para tener a mano el código y poder verlo en conjunto.


  \section{Nuevos requerimientos}
 
}

\title{Técnicas Avanzadas de Programación -- UTN -- FRBA\vspace{.2\baselineskip}
       \\ \cuatrimestre\
       \\ Ejercicio Integrador \
       \\ \nofinal{\large \medskip Versión \version\\}
       \bigskip
      }
\date{}
\newcommand{\newest}[1]{#1} %\textbf{Cursadas$\geq$2000:} #1 }
\newcommand{\oldest}[1]{}%{#1}

\newcommand{\flecha}{->}
\newcommand{\newconcept}[1]{\emph{#1}}

\DefineShortVerb{\|}

\author{}
	

\begin{document}
\maketitle

\section{Objetivos y recomendaciones}

\begin{itemize}
  
  \item Desarrollar un framework en objetos de complejidad media.
  \item La aplicación se debe construir en Ruby.
  \item La solución debe presentar buenas prácticas de diseño, tales como
  evitar repetición de código, buena distribución de responsabilidades, manejo
  adecuado de excepciones, etc.
  \item Este ejercicio se desarrollará los días 12 y 19 de Septiembre en clase. 
  El objetivo es poder desarrollar un ejercicio al nivel de dificultad de un 
  trabajo práctico en clase aplicando los conceptos aprendidos durante el cuatrimestre sobre metaprogramación.

\end{itemize}
    
    
\section{Introducción}

En Ruby existen, por defecto, dos formas de definir y compartir comportamiento entre los distintos objetos de nuestro sistema: Clases y Mixins (modules). Pero debido a sus características dinámicas y a su metamodelo maleable, es posible alterar las formas de recibir comportamiento e incluso agregar nuevas.

El objetivo de este trabajo práctico es, entonces, implementar y analizar una nueva forma de definir y compartir comportamiento en Ruby, puntualmente a través de \textbf{prototipos}.


\section{Comportamiento esperado}
Los requerimientos de cómo debe comportarse el nuevo sistema serán proporcionados en forma de código. La intención es que lo que se implemente pueda funcionar, pero no se busca acotar las posibles implementaciones. Luego, cada grupo deberá analizar en detalle las características particulares de su implementación, mostrando cómo se comporta en casos de conflictos y en relación con los elementos del metamodelo existentes.

El comportamiento esperado es el requerimiento definido, y debe estar testeado con pruebas unitarias automáticas, pero \textbf{no debe probarse sólo eso}.

\section{Partes}
El trabajo práctico está dividido en partes. Las partes están pensadas para guiarlos en el desarrollo, teniendo en cuenta la planificación de la materia. También cada parte va elaborando sobre la anterior, por lo que es recomendable ir resolviendo los puntos en el orden propuesto. Si bien lo más importante es el resultado final al momento de la entrega, \textbf{es preciso ir elaborando el TP a medida que es posible, para que no se junte mucha carga cerca de la fecha de entrega}.


\subsection{Prototipos programáticos}
Para esta primera parte, se busca introducir la nueva forma de definir comportamiento, pero sin azúcar sintáctico. La intención es poder hacer algo como lo que sigue:
\newpage 
 \begin{lstlisting}[language=Ruby]
	guerrero = PrototypedObject.new

	guerrero.set_property(:energia, 100)
	expect(guerrero.energia).to eq(100)

	guerrero.set_property(:potencial_defensivo, 10)
	guerrero.set_property(:potencial_ofensivo, 30)

	guerrero.set_method(:atacar_a, 
	  proc {
		|otro_guerrero|
		if(otro_guerrero.potencial_defensivo < self.potencial_ofensivo)
		  otro_guerrero.recibe_danio(self.potencial_ofensivo - otro_guerrero.potencial_defensivo)
		end
	  });

	guerrero.set_method(:recibe_danio, proc {...})

	otro_guerrero = guerrero.clone #clone es un metodo que ya viene definido en Ruby

	guerrero.atacar_a otro_guerrero

	expect(otro_guerrero.energia).to eq(80)	
  \end{lstlisting}
  
Hasta aquí el único agregado es la posibilidad de definir métodos y propiedades en cualquier objeto prototipado, sin necesidad que ese comportamiento provenga de una clase en particular.

\bigskip

Lo interesante viene ahora:
   \begin{lstlisting}[language=Ruby]
	espadachin = PrototypedObject.new

	espadachin.set_prototype(guerrero)

	espadachin.set_property(:habilidad, 0.5)
	espadachin.set_property(:potencial_espada, 30)

	espadachin.energia = 100

	{...} #mas inicializaciones
	#deberia llamar a super, pero eso lo resolvemos mas adelante
	espadachin.set_method(:potencial_ofensivo, proc {
	  @potencial_ofensivo + self.potencial_espada * self.habilidad
	})

	espadachin.atacar_a(otro_guerrero)
	expect(otro_guerrero.energia).to eq(75)
  \end{lstlisting}

Más interesante todavía es la relación que se crea entre el espadachin y su prototipo:
   \begin{lstlisting}[language=Ruby]
	guerrero.set_method(:sanar, proc {
	  self.energia = self.energia + 10                                
	})                                

	espadachin.sanar
	expect(espadachin.energia).to eq(110)
  \end{lstlisting}

Es decir, cualquier cambio en el prototipo impacta también en los objetos derivados de él. Cabe aclarar que el prototipo provee métodos y no estado.

\bigskip

Distinto sucede con la clonación: Los cambios no se impactan en el objeto que es clon:
\newpage
   \begin{lstlisting}[language=Ruby]
	expect {otro_guerrero.sanar}.to raise_error(NoMethodError)
  \end{lstlisting}
  Además, con la clonación se copia el estado del objeto, cosa que no debe suceder con los prototipos.

\bigskip

Tampoco son afectados los métodos que fueron redefinidos por el objeto derivado:
  
 \begin{lstlisting}[language=Ruby]
	guerrero.set_method(:potencial_ofensivo, proc {
	  1000
	})

	expect(espadachin.potencial_ofensivo).to eq(45)
  \end{lstlisting}
  
  Finalmente, también nos interesa hacer que el comportamiento “prototipable” sea algo que yo pueda usar en cualquier lado. Por ejemplo:
 \begin{lstlisting}[language=Ruby]
	class Object
	  include Prototyped
	end
  \end{lstlisting}
\subsection{Constructores}
Las clases proveen, además de comportamiento, un mecanismo para construir los objetos de manera tal que cada uno pueda tener inicializado su estado para funcionar según corresponda.

\bigskip

En principio, queremos un objeto o función que nos permita construir un objeto en base a un prototipo, inicializando su estado con parámetros:

\begin{lstlisting}[language=Ruby]
	Guerrero = PrototypedConstructor.new(guerrero, proc {
	 |guerrero_nuevo, una_energia, un_potencial_ofensivo, un_potencial_defensivo|
		guerrero_nuevo.energia = una_energia
		guerrero_nuevo.potencial_ofensivo = un_potencial_ofensivo
		guerrero_nuevo.potencial_defensivo = un_potencial_defensivo
	})                                                          

	un_guerrero = Guerrero.new(100, 30, 10)
	expect(un_guerrero.energia).to eq(100)
\end{lstlisting}
  
  El resultado del new sería un objeto con “guerrero” como prototipo y el estado inicializado como indica el proc que se pasa como parámetro.

\bigskip

Pero esa forma de definir los constructores requiere mucho trabajo. Luego veremos cómo hacerla más linda y usable, pero por lo pronto lo que podemos hacer es definir un constructor sencillo por defecto. Tenemos la información de las propiedades del prototipo para trabajar. Por ejemplo:


\begin{lstlisting}[language=Ruby]
	Guerrero = PrototypedConstructor.new(guerrero)

	un_guerrero = Guerrero.new(
	  {energia: 100, potencial_ofensivo: 30, potencial_defensivo: 10}
	)
	expect(un_guerrero.potencial_ofensivo).to eq(30)
\end{lstlisting}

Por el hecho de que no está claramente definido el orden de los parámetros, usamos un mapa.

\bigskip

Además, queremos definir un constructor que cree un prototipo y copie el estado actual del prototipo. Para ello, usamos lo siguiente:

\begin{lstlisting}[language=Ruby]
	Guerrero = PrototypedConstructor.copy(guerrero)

	un_guerrero = Guerrero.new
	expect(un_guerrero.potencial_defensivo).to eq(10)
\end{lstlisting}

Finalmente, con lo que hicimos también podemos producir un constructor que altere los métodos que entiende el objeto, produciendo un nuevo tipo de objeto. Nos interesa que un constructor pueda extender de otro:

\begin{lstlisting}[language=Ruby]
	#Guerrero es la primer variante de constructores, que recibe 3 parametros
	Espadachin = Guerrero.extended {
	 |espadachin, habilidad, potencial_espada|
		espadachin.set_property(:habilidad, habilidad)
		espadachin.set_property(:potencial_espada, potencial_espada)

		espadachin.set_method(:potencial_ofensivo, proc {
		  @potencial_ofensivo + self.potencial_espada * self.habilidad
		})

	})                                                          

	espadachin = Espadachin.new(100, 30, 10, 0.5, 30)
	expect(espadachin.potencial_ofensivo).to eq(45)
\end{lstlisting}

La convención que se define (de momento) es que los parámetros del constructor se pasan y se consumen en orden. Los primeros 3 parámetros del constructor generado van para Guerrero, porque Guerrero recibe 3 parámetros en su constructor. Si fuera la segunda variante del constructor de guerrero, el constructor de Espadachín debería llamarse así:

\begin{lstlisting}[language=Ruby]
	espadachin = Espadachin.new({energia: 100, potencial_ofensivo: 30, potencial_defensivo: 10}, 0.5, 30)
\end{lstlisting}



\subsection{Azúcar sintáctico}
La implementación de prototipos que hicimos funciona, pero tiene problemas de expresividad, y realmente no es muy usable, muy cómoda para el programador. La intención ahora es entonces agregar una interfaz a nuestros objetos prototipados para que sean más fácilmente utilizables, aprovechando las características de Ruby.

\bigskip

Comenzamos con los elementos de la primera parte:
\begin{lstlisting}[language=Ruby]
	guerrero_proto = PrototypedObject.new
	guerrero_proto.energia = 100
	expect(guerrero_proto.energia).to eq(100)

	guerrero_proto.potencial_defensivo = 10
	guerrero_proto.potencial_ofensivo = 30

	guerrero_proto.atacar_a = proc { |otro_guerrero| ... });
	guerrero_proto.recibe_danio = proc {...}

	Guerrero = PrototypedConstructor.copy(guerrero_proto)

	un_guerrero = Guerrero.new
	Guerrero.new.atacar_a(un_guerrero)

	expect(un_guerrero.energia).to eq(80)
\end{lstlisting}
  
  La idea es que, dinámicamente, uno puede “asignar” propiedades y métodos a los objetos prototipados.

\bigskip

También es incómodo repetir el receptor del mensaje en todo momento. Podemos hacer algo mejor:
\newpage

\begin{lstlisting}[language=Ruby]
	guerrero_proto = PrototypedObject.new {
	  self.energia = 100
	  self.potencial_ofensivo = 30
	  self.potencial_defensivo = 10

	  self.atacar_a = proc { |otro_guerrero| ... });
	  self.recibe_danio = proc {...}
	}
\end{lstlisting}
  
  Podemos aplicar ideas similares para los constructores:
  
  
  
\begin{lstlisting}[language=Ruby]
	Guerrero = PrototypedConstructor.new(guerrero) do |una_energia, un_potencial_ofensivo, un_potencial_defensivo|
	  self.energia = una_energia
	  self.potencial_ofensivo = un_potencial_ofensivo
	  self.potencial_defensivo = un_potencial_defensivo
	end
\end{lstlisting}
También, puede ser práctico definir y utilizar el prototipo en el mismo momento:
\begin{lstlisting}[language=Ruby]
	Guerrero = PrototypedConstructor.create {
		self.atacar_a = proc { |otro_guerrero| ... });
		self.recibe_danio = proc {...}
	  }.with {
		|una_energia, un_potencial_ofensivo, un_potencial_defensivo|
		  self.energia = una_energia
		  self.potencial_ofensivo = un_potencial_ofensivo
		  self.potencial_defensivo = un_potencial_defensivo
	}
\end{lstlisting}

Es necesario pasarle el procedimiento del constructor para inicializar las propiedades porque no tenemos información de las propiedades existentes. También se podría definir un método para crear un constructor por defecto más sencillo:
\begin{lstlisting}[language=Ruby]
	Guerrero = PrototypedConstructor.create {
		self.atacar_a = proc { |otro_guerrero| ... });
		self.recibe_danio = proc {...}
	 }.with_properties([:energia, :potencial_ofensivo, :potencial_defensivo])
\end{lstlisting}
Si yo quisiera obtener el prototipo de un constructor de estos, debería poder pedírselo, y valen las mismas condiciones que para los prototipos con nombre:
\begin{lstlisting}[language=Ruby]
	atila = Guerrero.new(100, 50, 30)

	expect(atila.potencial_ofensivo).to eq(50)

	proto_guerrero = Guerrero.prototype
	proto_guerrero.potencial_ofensivo = proc {
	   1000
	}

	expect(atila.potencial_ofensivo).to eq(1000)
\end{lstlisting}
\newpage
Finalmente, debería poder aprovechar esta nueva sintaxis también para la extensión de constructores:
\begin{lstlisting}[language=Ruby]
	Espadachin = Guerrero.extended {
	  |una_habilidad, un_potencial_espada|
		self.habilidad = una_habilidad
		self.potencial_espada = un_potencial_espada

		self.potencial_ofensivo = proc {
		  @potencial_ofensivo + self.potencial_espada * self.habilidad
		}
	}
\end{lstlisting}


\subsection{Llamar al prototipo anterior y Múltiples prototipos}
Anteriormente, cuando definimos Espadachin nos encontramos con el inconveniente de que queríamos utilizar la implementación previa de potencial\_ofensivo (que venía de guerrero), pero no teníamos forma de acceder a ella, porque estábamos pisando ese método.

\bigskip

Nos interesa entonces definir un mecanismo para poder llamar a la implementación siguiente de la cadena de prototipos:
\begin{lstlisting}[language=Ruby]
	Espadachin = Guerrero.extended {
	  |una_habilidad, un_potencial_espada|
		self.habilidad = una_habilidad
		self.potencial_espada = un_potencial_espada

		self.potencial_ofensivo = proc {
		  call_next + self.potencial_espada * self.habilidad
		}
	}
\end{lstlisting}
Finalmente, nos interesa incorporar una idea similar a lo que hacemos con Mixins o con herencia múltiple y permitir múltiples prototipos en un mismo objeto:
\begin{lstlisting}[language=Ruby]
	Guerrero.prototype.set_prototypes([proto_atacante, proto_defensor])
\end{lstlisting}
Cabe aclarar que proto\_atacante y proto\_defensor no deben verse afectados en su cadena de prototipado: No es válido que proto\_defensor pase a ser prototipo de proto\_atacante ni vice versa.

\bigskip

Queda a criterio del grupo cómo se efectúa la resolución de conflictos en caso que existieran y cómo funciona el call\_next al tener múltiples prototipos.

\end{document}
